\documentclass{ximera}

%\usepackage{todonotes}

\newcommand{\todo}{}

\usepackage{tkz-euclide}
\tikzset{>=stealth} %% cool arrow head
\tikzset{shorten <>/.style={ shorten >=#1, shorten <=#1 } } %% allows shorter vectors

\usetikzlibrary{backgrounds} %% for boxes around graphs
\usetikzlibrary{shapes,positioning}  %% Clouds and stars
\usetikzlibrary{matrix} %% for matrix
\usepgfplotslibrary{polar} %% for polar plots
\usetkzobj{all}
\usepackage[makeroom]{cancel} %% for strike outs
%\usepackage{mathtools} %% for pretty underbrace % Breaks Ximera
\usepackage{multicol}





\usepackage{array}
\setlength{\extrarowheight}{+.1cm}   
\newdimen\digitwidth
\settowidth\digitwidth{9}
\def\divrule#1#2{
\noalign{\moveright#1\digitwidth
\vbox{\hrule width#2\digitwidth}}}





\newcommand{\RR}{\mathbb R}
\newcommand{\R}{\mathbb R}
\newcommand{\N}{\mathbb N}
\newcommand{\Z}{\mathbb Z}

%\renewcommand{\d}{\,d\!}
\renewcommand{\d}{\mathop{}\!d}
\newcommand{\dd}[2][]{\frac{\d #1}{\d #2}}
\newcommand{\pp}[2][]{\frac{\partial #1}{\partial #2}}
\renewcommand{\l}{\ell}
\newcommand{\ddx}{\frac{d}{\d x}}

\newcommand{\zeroOverZero}{\ensuremath{\boldsymbol{\tfrac{0}{0}}}}
\newcommand{\inftyOverInfty}{\ensuremath{\boldsymbol{\tfrac{\infty}{\infty}}}}
\newcommand{\zeroOverInfty}{\ensuremath{\boldsymbol{\tfrac{0}{\infty}}}}
\newcommand{\zeroTimesInfty}{\ensuremath{\small\boldsymbol{0\cdot \infty}}}
\newcommand{\inftyMinusInfty}{\ensuremath{\small\boldsymbol{\infty - \infty}}}
\newcommand{\oneToInfty}{\ensuremath{\boldsymbol{1^\infty}}}
\newcommand{\zeroToZero}{\ensuremath{\boldsymbol{0^0}}}
\newcommand{\inftyToZero}{\ensuremath{\boldsymbol{\infty^0}}}



\newcommand{\numOverZero}{\ensuremath{\boldsymbol{\tfrac{\#}{0}}}}
\newcommand{\dfn}{\textbf}
%\newcommand{\unit}{\,\mathrm}
\newcommand{\unit}{\mathop{}\!\mathrm}
\newcommand{\eval}[1]{\bigg[ #1 \bigg]}
\newcommand{\seq}[1]{\left( #1 \right)}
\renewcommand{\epsilon}{\varepsilon}
\renewcommand{\iff}{\Leftrightarrow}

\DeclareMathOperator{\arccot}{arccot}
\DeclareMathOperator{\arcsec}{arcsec}
\DeclareMathOperator{\arccsc}{arccsc}
\DeclareMathOperator{\si}{Si}
\DeclareMathOperator{\proj}{proj}
\DeclareMathOperator{\scal}{scal}


\newcommand{\tightoverset}[2]{% for arrow vec
  \mathop{#2}\limits^{\vbox to -.5ex{\kern-0.75ex\hbox{$#1$}\vss}}}
\newcommand{\arrowvec}[1]{\tightoverset{\scriptstyle\rightharpoonup}{#1}}
\renewcommand{\vec}{\mathbf}
\newcommand{\veci}{\vec{i}}
\newcommand{\vecj}{\vec{j}}
\newcommand{\veck}{\vec{k}}
\newcommand{\vecl}{\boldsymbol{\l}}

\newcommand{\dotp}{\bullet}
\newcommand{\cross}{\boldsymbol\times}
\newcommand{\grad}{\boldsymbol\nabla}
\newcommand{\divergence}{\grad\dotp}
\newcommand{\curl}{\grad\cross}
%\DeclareMathOperator{\divergence}{divergence}
%\DeclareMathOperator{\curl}[1]{\grad\cross #1}


\colorlet{textColor}{black} 
\colorlet{background}{white}
\colorlet{penColor}{blue!50!black} % Color of a curve in a plot
\colorlet{penColor2}{red!50!black}% Color of a curve in a plot
\colorlet{penColor3}{red!50!blue} % Color of a curve in a plot
\colorlet{penColor4}{green!50!black} % Color of a curve in a plot
\colorlet{penColor5}{orange!80!black} % Color of a curve in a plot
\colorlet{fill1}{penColor!20} % Color of fill in a plot
\colorlet{fill2}{penColor2!20} % Color of fill in a plot
\colorlet{fillp}{fill1} % Color of positive area
\colorlet{filln}{penColor2!20} % Color of negative area
\colorlet{fill3}{penColor3!20} % Fill
\colorlet{fill4}{penColor4!20} % Fill
\colorlet{fill5}{penColor5!20} % Fill
\colorlet{gridColor}{gray!50} % Color of grid in a plot

\newcommand{\surfaceColor}{violet}
\newcommand{\surfaceColorTwo}{redyellow}
\newcommand{\sliceColor}{greenyellow}




\pgfmathdeclarefunction{gauss}{2}{% gives gaussian
  \pgfmathparse{1/(#2*sqrt(2*pi))*exp(-((x-#1)^2)/(2*#2^2))}%
}


%%%%%%%%%%%%%
%% Vectors
%%%%%%%%%%%%%

%% Simple horiz vectors
\renewcommand{\vector}[1]{\left\langle #1\right\rangle}


%% %% Complex Horiz Vectors with angle brackets
%% \makeatletter
%% \renewcommand{\vector}[2][ , ]{\left\langle%
%%   \def\nextitem{\def\nextitem{#1}}%
%%   \@for \el:=#2\do{\nextitem\el}\right\rangle%
%% }
%% \makeatother

%% %% Vertical Vectors
%% \def\vector#1{\begin{bmatrix}\vecListA#1,,\end{bmatrix}}
%% \def\vecListA#1,{\if,#1,\else #1\cr \expandafter \vecListA \fi}

%%%%%%%%%%%%%
%% End of vectors
%%%%%%%%%%%%%

%\newcommand{\fullwidth}{}
%\newcommand{\normalwidth}{}



%% makes a snazzy t-chart for evaluating functions
%\newenvironment{tchart}{\rowcolors{2}{}{background!90!textColor}\array}{\endarray}

%%This is to help with formatting on future title pages.
\newenvironment{sectionOutcomes}{}{} 



%% Flowchart stuff
%\tikzstyle{startstop} = [rectangle, rounded corners, minimum width=3cm, minimum height=1cm,text centered, draw=black]
%\tikzstyle{question} = [rectangle, minimum width=3cm, minimum height=1cm, text centered, draw=black]
%\tikzstyle{decision} = [trapezium, trapezium left angle=70, trapezium right angle=110, minimum width=3cm, minimum height=1cm, text centered, draw=black]
%\tikzstyle{question} = [rectangle, rounded corners, minimum width=3cm, minimum height=1cm,text centered, draw=black]
%\tikzstyle{process} = [rectangle, minimum width=3cm, minimum height=1cm, text centered, draw=black]
%\tikzstyle{decision} = [trapezium, trapezium left angle=70, trapezium right angle=110, minimum width=3cm, minimum height=1cm, text centered, draw=black]


\title[Dig-In]{Concavity and the Second Derivative Test}


\outcome{State the Second Derivative Test.}
\outcome{Apply the Second Derivative Test.}
\outcome{Define inflection points.}
\outcome{Find inflection points.}
  


\begin{document}
\begin{abstract}
We use second derivatives to help locate extrema.  
\end{abstract}
\maketitle




%%CONCAVITY%%

\section{Concavity}
\begin{abstract}
  Here we examine what the second derivative tells us about the
  geometry of functions.
\end{abstract}

We know that the sign of the derivative tells us whether a function is
increasing or decreasing at some point. Likewise, the sign of the
second derivative $f''(x)$ tells us whether $f'(x)$ is increasing or
decreasing at $x$. We summarize the consequences of this idea in the table below:

\begin{image}
  \begin{tikzpicture}
    \draw (0,0) -- (0,12);
    \draw (0,0) -- (12,0);
    \draw (6,0) -- (6,12);
    \draw (0,6) -- (12,6);
    \draw (12,0) -- (12,12);
    \draw (0,12) -- (12,12);
    
    \node at (-1.3,9) {\Large$0<f''(x)$};
    \node at (-1.3,3) {\Large$f''(x)<0$};
    \node at (3,12.4) {\Large$f'(x)<0$};
    \node at (9,12.4) {\Large$0<f'(x)$};
    
    \draw [penColor,ultra thick,domain=180:270] plot ({2*cos(\x)+4}, {2*sin(\x)+11});
    \draw [penColor,ultra thick,domain=270:360] plot ({2*cos(\x)+8}, {2*sin(\x)+11});
    \draw [penColor,ultra thick,domain=0:90] plot ({2*cos(\x)+2}, {2*sin(\x)+3});
    \draw [penColor,ultra thick,domain=180:90] plot ({2*cos(\x)+10}, {2*sin(\x)+3});

    \node at (3,7.5) [text width=5cm] {\large
      Here $y=f(x)$ is decreasing, while the rate itself is increasing.
      In this case the curve is \dfn{concave up}.};

    \node at (9,7.5) [text width=5cm] {\large
      Here $y=f(x)$ is increasing, while the rate itself is increasing.
      In this case the curve is \dfn{concave up}.};

    \node at (3,1.5) [text width=5cm] {\large
      Here $y=f(x)$ is decreasing, while the rate itself is decreasing.
      In this case the curve is \dfn{concave down}.};

    \node at (9,1.5) [text width=5cm] {\large
      Here $y=f(x)$ is increasing, while the rate itself is decreasing.
      In this case the curve is \dfn{concave down}.};
  \end{tikzpicture}
\end{image}

If we are trying to understand the shape of the graph of a function,
knowing where it is concave up and concave down helps us to get a more
accurate picture. It is worth summarizing what we have seen already in
to a single theorem.

\begin{theorem}[Test for Concavity]\index{concavity test}
Suppose that $f''(x)$ exists on an interval.
\begin{enumerate}
\item $f''(x)>0$ on that interval whenever $y=f(x)$ is concave up on that interval.
\item $f''(x)<0$ on that interval whenever $y=f(x)$ is concave down on that interval.
\end{enumerate}
\end{theorem}


\begin{example}
  Let $f$ be a continuous function and suppose that:
  \begin{itemize}
  \item $f'(x) > 0$ for $-1< x<1$.
  \item $f'(x) < 0$ for $-2< x<-1$ and $1<x<2$.
  \item $f''(x) > 0$ for $-2<x<0$ and $1<x< 2$.
  \item $f''(x) < 0$ for $0<x< 1$.  
  \end{itemize}
  Sketch a possible graph of $f$.
  \begin{explanation}
    Start by marking where the derivative changes sign and indicate
    intervals where $f$ is increasing and intervals $f$ is
    decreasing. The function $f$ has a negative derivative from $-2$
    to $x=\answer[given]{-1}$. This means that $f$ is
    \wordChoice{\choice{increasing}\choice[correct]{decreasing}} on
    this interval. The function $f$ has a positive derivative from
    $x=\answer[given]{-1}$ to $x=\answer[given]{1}$. This means that
    $f$ is
    \wordChoice{\choice[correct]{increasing}\choice{decreasing}} on
    this interval. Finally, The function $f$ has a negative derivative
    from $x=\answer[given]{1}$ to $2$. This means that $f$ is
    \wordChoice{\choice{increasing}\choice[correct]{decreasing}} on
    this interval.
  \begin{image}
    \begin{tikzpicture}
    \begin{axis}[
        xmin=-2,xmax=2,ymin=-2,ymax=2,
        axis lines=center,
        width=6in,
        height=3in,
        every axis y label/.style={at=(current axis.above origin),anchor=south},
        every axis x label/.style={at=(current axis.right of origin),anchor=west},
      ]
      \addplot [dashed, penColor2] plot coordinates {(-1,-2) (-1,2)}; %% Critical points
      \addplot [dashed, penColor2] plot coordinates {(1,-2) (1,2)}; %% Critical points

      \addplot [->, line width=10, penColor!10!background] plot coordinates {(-1+.2,-2+.2) (1-.2,2-.2)};
      \addplot [->, line width=10, penColor!10!background] plot coordinates {(-2+.2,2-.2) (-1-.2,-2+.2)};
      \addplot [->, line width=10, penColor!10!background] plot coordinates {(1+.2,2-.2) (2-.2,-2+.2)}; 
      
      %\addplot [very thick,penColor,smooth, domain=(-2:2)] {x^3+x^2-2*x)};
    \end{axis}
  \end{tikzpicture}
  \end{image}
  Now we should sketch the concavity: \wordChoice{\choice[correct]{concave up}\choice{concave down}} when the second
  derivative is positive, \wordChoice{\choice{concave up}\choice[correct]{concave down}} when the second derivative is
  negative.
    \begin{image}
    \begin{tikzpicture}
    \begin{axis}[
        xmin=-2,xmax=2,ymin=-2,ymax=2,
        axis lines=center,
        width=6in,
        height=3in,
        every axis y label/.style={at=(current axis.above origin),anchor=south},
        every axis x label/.style={at=(current axis.right of origin),anchor=west},
      ]
      \addplot [dashed, penColor2] plot coordinates {(-1,-2) (-1,2)}; %% Critical points
      \addplot [dashed, penColor2] plot coordinates {(1,-2) (1,2)}; %% Critical points

      %\addplot [->, line width=10, penColor!10!background] plot coordinates {(-1+.2,-2+.2) (1-.2,2-.2)};
      %\addplot [->, line width=10, penColor!10!background] plot coordinates {(-2+.2,2-.2) (-1-.2,-2+.2)};
      %\addplot [->, line width=10, penColor!10!background] plot coordinates {(1+.2,2-.2) (2-.2,-2+.2)};

      \addplot [penColor3!20!background,line width=10,domain=180:270] ({-1.1+.7*cos(x)}, {1.2+2*sin(x)});
      \addplot [penColor3!20!background,line width=10,domain=270:360] ({-.9+.7*cos(x)}, {-.1+.7*sin(x)});
      \addplot [penColor3!20!background,line width=10,domain=180:90] ({.9+.7*cos(x)}, {.1+.7*sin(x)});
      \addplot [penColor3!20!background,line width=10,domain=180:265] ({1.9+.7*cos(x)}, {.9+ 2*sin(x)});

      \addplot [->, line width=10, penColor3!20!background] plot coordinates {(-1.1,-.1-.7) (-1,-.1-.7)};
      \addplot [->, line width=10, penColor3!20!background] plot coordinates {(.9,.1+.7) (1,.1+.7)};
      
      \addplot [->, line width=10, penColor3!20!background] plot coordinates {(-.9+.7,-.2) (-.9+.8,.2)};
      \addplot [->, line width=10, penColor3!20!background] plot coordinates {(1.8,-1.1) (2,-1.2)};
      
      %\addplot [very thick,penColor,smooth, domain=(-2:2)] {x^3+x^2-2*x)};
    \end{axis}
  \end{tikzpicture}
    \end{image}
    Finally, we can sketch our curve:
        \begin{image}
    \begin{tikzpicture}
    \begin{axis}[
        xmin=-2,xmax=2,ymin=-2,ymax=2,
        axis lines=center,
        width=6in,
        height=3in,
        every axis y label/.style={at=(current axis.above origin),anchor=south},
        every axis x label/.style={at=(current axis.right of origin),anchor=west},
      ]
      \addplot [dashed, penColor2] plot coordinates {(-1,-2) (-1,2)}; %% Critical points
      \addplot [dashed, penColor2] plot coordinates {(1,-2) (1,2)}; %% Critical points

      \addplot [penColor,ultra thick,domain=-2:1,smooth] {(-x^3+3*x)*.5};
      \addplot [penColor,ultra thick,domain=1:2,smooth] {(-(x-3)^3+3*(x-3))*.5};
    \end{axis}
  \end{tikzpicture}
  \end{image}
  \end{explanation}
\end{example}


%%INFLECTION POINTS%%
\section{Inflection points}


Recall that when looking for intervals of increase or decrease we started our search by looking where the function was neither increasing nor decreasing.  We can apply this same logic to intervals of concavity. Once we have intervals of concave up and concave down, we can analyze any points where the concavity does change. 


\begin{definition}\index{inflection point}
If $f$ is continuous and its concavity changes either from up to down
or down to up at $x=a$, then $f$ has an \dfn{inflection point} at
$x=a$.
\end{definition}

It is instructive to see some examples of inflection points:
\begin{image}
\begin{tikzpicture}
	\begin{axis}[
            %height=7cm,
            %width=2in,
            width=6in,
            height=2in,
            %ymax=8,
            %ymin=-1,
            axis lines=none,
            clip=false,
          ]
          \addplot [very thick, penColor, smooth, domain=(0:1)] {(x-1)^2+1};
          \addplot [very thick, penColor, smooth, domain=(1:2)] {-(x-1)^2+1};
          \addplot[color=penColor,fill=penColor,only marks,mark=*] coordinates{(1,1)};
          \node at (axis cs:1,-.5) [text width=2in] {This is an inflection point. The concavity changes from concave up to concave down.};
        
          \addplot [very thick, penColor, smooth,domain=(4:5)] {-sqrt(abs(1-(x-4)))+1};
          \addplot [very thick, penColor, smooth,domain=(5:6)] {sqrt((x-4)-1)+1};
          \addplot[color=penColor,fill=penColor,only marks,mark=*] coordinates{(5,1)};
          \node at (axis cs:5,-.5) [text width=2in] {This is an inflection point. The concavity changes from concave up to concave down.};
        \end{axis}
\end{tikzpicture}
\end{image}

It is also instructive to see some nonexamples of inflection points:
\begin{image}
\begin{tikzpicture}
	\begin{axis}[
            %height=7cm,
            %width=2in,
            width=6in,
            height=2in,
            %ymax=8,
            %ymin=-1,
            axis lines=none,
            clip=false,
          ]
          \addplot [very thick, penColor2, smooth, domain=(0:2)] {-(x-1)^2+1};
          \addplot[color=penColor2,fill=penColor2,only marks,mark=*] coordinates{(1,1)};
          \node at (axis cs:1,-.5) [text width=2in] {This is \textbf{not} an inflection point. The curve is concave down on either side of the point.};
          \addplot [very thick, penColor2, smooth,domain=(4:5)] {sqrt(abs(1-(x-4)))};
          \addplot [very thick, penColor2, smooth,domain=(5:6)] {sqrt(x-5)};
          \addplot[color=penColor2,fill=penColor2,only marks,mark=*] coordinates{(5,0)};
          \node at (axis cs:5,-.5) [text width=2in] {This is \textbf{not} an inflection point. The curve is concave down on either side of the point.};
        \end{axis}
\end{tikzpicture}
\end{image}

We identify inflection points by first finding $x$ such that $f''(x)$
is zero or undefined and then checking to see whether $f''(x)$ does in
fact go from positive to negative or negative to positive at these
points.

\begin{warning}
Even if $f''(a) = 0$, the point determined by $x=a$ might \textbf{not}
be an inflection point.
\end{warning}


\begin{example}
Describe the concavity of $f(x)=x^3-x$. 

\begin{explanation}
To start, compute the first and second derivative of $f(x)$ with
respect to $x$,
\[
f'(x)=\answer[given]{3x^2-1}\qquad\text{and}\qquad f''(x)=\answer[given]{6x}.
\]
Since $f''(0)=0$, there is potentially an inflection point at
$x=0$. Using test points, we note the concavity does change from down
to up, hence there is an inflection point at $x=0$. The curve is
concave down for all $x<0$ and concave up for all $x>0$, see the
graphs of $f(x) = x^3-x$ and $f''(x) = 6x$.
\begin{image}
\begin{tikzpicture}
	\begin{axis}[
            domain=-3:3,
            ymax=3,
            ymin=-3,
            axis lines =middle, xlabel=$x$, ylabel=$y$,
            every axis y label/.style={at=(current axis.above origin),anchor=south},
            every axis x label/.style={at=(current axis.right of origin),anchor=west}
          ]
          \addplot [very thick, penColor, smooth] {x^3-x};
          \addplot [very thick, penColor4, smooth] {6*x};         
          \node at (axis cs:-.75,.6) [anchor=west] {\color{penColor}$f$};  
          \node at (axis cs:.2,1) [anchor=west] {\color{penColor4}$f''$};
          \addplot[color=penColor4!50!penColor,fill=penColor4!50!penColor,only marks,mark=*] coordinates{(0,0)};  %% closed hole
        \end{axis}
\end{tikzpicture}
%% \caption{A plot of $f(x) = x^3-x$ and $f''(x) = 6x$. We can see that
%%   the concavity change at $x=0$.}
%% \label{figure:3x^2-1}
%% \end{marginfigure}
\end{image}
\end{explanation}
\end{example}


Note that we need to compute and analyze the second derivative to
understand concavity, so we may as well try to use the second
derivative test for maxima and minima. If for some reason this fails
we can then try one of the other tests.

\section{The second derivative test}


Recall the first derivative test:
\begin{itemize}
\item If $f'(x)>0$ to the left of $a$ and $f'(x)<0$ to the right of
  $a$, then $f(a)$ is a local maximum.
\item If $f'(x)<0$ to the left of $a$ and $f'(x)>0$ to the right of
  $a$, then $f(a)$ is a local minimum.
\end{itemize}

If $f'$ changes from positive to negative it is decreasing. In this
case, $f''$ might be negative, and if in fact $f''$ is negative
then $f'$ is definitely decreasing, so there is a local maximum at
the point in question. On the other hand, if $f'$ changes from
negative to positive it is increasing. Again, this means that
$f''$ might be positive, and if in fact $f''$ is positive then
$f'$ is definitely increasing, so there is a local minimum at the
point in question. We summarize this as the \textit{second derivative
  test}.

\begin{theorem}[Second Derivative Test]\index{second derivative test}\label{T:sdt}
Suppose that $f''(x)$ is continuous on an open interval and that
$f'(a)=0$ for some value of $a$ in that interval.
\begin{itemize}
\item If $f''(a) <0$, then $f$ has a local maximum at $a$.
\item If $f''(a) >0$, then $f$ has a local minimum at $a$.
\item If $f''(a) =0$, then the test is inconclusive. In this case,
  $f$ may or may not have a local extremum at $x=a$.
\end{itemize}
\end{theorem}


The second derivative test is often the easiest way to identify local
maximum and minimum points. Sometimes the test fails and sometimes
the second derivative is quite difficult to evaluate. In such cases we
must fall back on one of the previous tests.

\begin{example}
Once again, consider the function 
\[
f(x) = \frac{x^4}{4}+\frac{x^3}{3}-x^2
\]
Use the second derivative test, to locate the
local extrema of $f$.

\begin{explanation}
Start by computing
\[
f'(x) = \answer[given]{x^3 + x^2 -2x} \qquad\text{and}\qquad f''(x) = \answer[given]{3x^2 + 2x-2}.
\] 
Using the same technique as we used before, we find that 
\[
f'(-2) = \answer[given]{0},\qquad f'(0) = \answer[given]{0}, \qquad f'(1) = \answer[given]{0}. 
\]
Now we'll attempt to use the second derivative test,
\[
f''(-2) = \answer[given]{6}, \qquad f''(0) =\answer[given]{ -2}, \qquad f''(1) = \answer[given]{3}.
\]
Hence we see that $f$ has a local minimum at $x=-2$, a local
maximum at $x=0$, and a local minimum at $x=1$, see below for a plot
of $f(x) =x^4/4 + x^3/3 -x^2$ and $f''(x) = 3x^2 + 2x -2$:
\begin{image}
\begin{tikzpicture}
	\begin{axis}[
            domain=-4:4,
            ymax=7,
            ymin=-4,
            %samples=100,
            axis lines =middle, xlabel=$x$, ylabel=$y$,
            every axis y label/.style={at=(current axis.above origin),anchor=south},
            every axis x label/.style={at=(current axis.right of origin),anchor=west}
          ]
          \addplot [dashed, textColor, smooth] plot coordinates {(-2,-2.667) (-2,6)}; %% {.451};
          \addplot [dashed, textColor, smooth] plot coordinates {(1,0) (1,3)}; %% axis{2.215};

          \addplot [very thick, penColor, smooth] {(x^4)/4 + (x^3)/3 -x^2};
          \addplot [very thick, penColor4, smooth] {3*x^2 + 2*x -2};

          \node at (axis cs:-1.7,-2.7) [anchor=west] {\color{penColor}$f$};  
          \node at (axis cs:-1.5,2) [anchor=west] {\color{penColor4}$f''$};

          \addplot[color=penColor4,fill=penColor4,only marks,mark=*] coordinates{(-2,6)};  %% closed hole
          \addplot[color=penColor4,fill=penColor4,only marks,mark=*] coordinates{(1,3)};  %% closed hole
          \addplot[color=penColor4,fill=penColor4,only marks,mark=*] coordinates{(0,-2)};  %% closed hole
          \addplot[color=penColor,fill=penColor,only marks,mark=*] coordinates{(0,0)};  %% closed hole
          \addplot[color=penColor,fill=penColor,only marks,mark=*] coordinates{(-2,.-2.667)};  %% closed hole
          \addplot[color=penColor,fill=penColor,only marks,mark=*] coordinates{(1,-.4167)};  %% closed hole
        \end{axis}
\end{tikzpicture}
%% \caption{A plot of $f(x) =x^4/4 + x^3/3 -x^2$ and $f''(x) = 3x^2 + 2x -2$.}
%% \label{figure:SDT(x^4)/4 + (x^3)/3 -x^2}
\end{image}

\end{explanation}
\end{example}


\begin{problem}
  If $f''(a)=0$, what does the second derivative test tell us?
  \begin{multipleChoice}
    \choice{The function has a local extrema at $x=a$.}
    \choice{The function does not have a local extrema at $x=a$.}
    \choice[correct]{It gives no information on whether $x=a$ is a local extremum.} 
  \end{multipleChoice}
  
\end{problem}


\end{document}
