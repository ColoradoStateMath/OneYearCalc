\documentclass{ximera}

%\usepackage{todonotes}

\newcommand{\todo}{}

\usepackage{tkz-euclide}
\tikzset{>=stealth} %% cool arrow head
\tikzset{shorten <>/.style={ shorten >=#1, shorten <=#1 } } %% allows shorter vectors

\usetikzlibrary{backgrounds} %% for boxes around graphs
\usetikzlibrary{shapes,positioning}  %% Clouds and stars
\usetikzlibrary{matrix} %% for matrix
\usepgfplotslibrary{polar} %% for polar plots
\usetkzobj{all}
\usepackage[makeroom]{cancel} %% for strike outs
%\usepackage{mathtools} %% for pretty underbrace % Breaks Ximera
\usepackage{multicol}





\usepackage{array}
\setlength{\extrarowheight}{+.1cm}   
\newdimen\digitwidth
\settowidth\digitwidth{9}
\def\divrule#1#2{
\noalign{\moveright#1\digitwidth
\vbox{\hrule width#2\digitwidth}}}





\newcommand{\RR}{\mathbb R}
\newcommand{\R}{\mathbb R}
\newcommand{\N}{\mathbb N}
\newcommand{\Z}{\mathbb Z}

%\renewcommand{\d}{\,d\!}
\renewcommand{\d}{\mathop{}\!d}
\newcommand{\dd}[2][]{\frac{\d #1}{\d #2}}
\newcommand{\pp}[2][]{\frac{\partial #1}{\partial #2}}
\renewcommand{\l}{\ell}
\newcommand{\ddx}{\frac{d}{\d x}}

\newcommand{\zeroOverZero}{\ensuremath{\boldsymbol{\tfrac{0}{0}}}}
\newcommand{\inftyOverInfty}{\ensuremath{\boldsymbol{\tfrac{\infty}{\infty}}}}
\newcommand{\zeroOverInfty}{\ensuremath{\boldsymbol{\tfrac{0}{\infty}}}}
\newcommand{\zeroTimesInfty}{\ensuremath{\small\boldsymbol{0\cdot \infty}}}
\newcommand{\inftyMinusInfty}{\ensuremath{\small\boldsymbol{\infty - \infty}}}
\newcommand{\oneToInfty}{\ensuremath{\boldsymbol{1^\infty}}}
\newcommand{\zeroToZero}{\ensuremath{\boldsymbol{0^0}}}
\newcommand{\inftyToZero}{\ensuremath{\boldsymbol{\infty^0}}}



\newcommand{\numOverZero}{\ensuremath{\boldsymbol{\tfrac{\#}{0}}}}
\newcommand{\dfn}{\textbf}
%\newcommand{\unit}{\,\mathrm}
\newcommand{\unit}{\mathop{}\!\mathrm}
\newcommand{\eval}[1]{\bigg[ #1 \bigg]}
\newcommand{\seq}[1]{\left( #1 \right)}
\renewcommand{\epsilon}{\varepsilon}
\renewcommand{\iff}{\Leftrightarrow}

\DeclareMathOperator{\arccot}{arccot}
\DeclareMathOperator{\arcsec}{arcsec}
\DeclareMathOperator{\arccsc}{arccsc}
\DeclareMathOperator{\si}{Si}
\DeclareMathOperator{\proj}{proj}
\DeclareMathOperator{\scal}{scal}


\newcommand{\tightoverset}[2]{% for arrow vec
  \mathop{#2}\limits^{\vbox to -.5ex{\kern-0.75ex\hbox{$#1$}\vss}}}
\newcommand{\arrowvec}[1]{\tightoverset{\scriptstyle\rightharpoonup}{#1}}
\renewcommand{\vec}{\mathbf}
\newcommand{\veci}{\vec{i}}
\newcommand{\vecj}{\vec{j}}
\newcommand{\veck}{\vec{k}}
\newcommand{\vecl}{\boldsymbol{\l}}

\newcommand{\dotp}{\bullet}
\newcommand{\cross}{\boldsymbol\times}
\newcommand{\grad}{\boldsymbol\nabla}
\newcommand{\divergence}{\grad\dotp}
\newcommand{\curl}{\grad\cross}
%\DeclareMathOperator{\divergence}{divergence}
%\DeclareMathOperator{\curl}[1]{\grad\cross #1}


\colorlet{textColor}{black} 
\colorlet{background}{white}
\colorlet{penColor}{blue!50!black} % Color of a curve in a plot
\colorlet{penColor2}{red!50!black}% Color of a curve in a plot
\colorlet{penColor3}{red!50!blue} % Color of a curve in a plot
\colorlet{penColor4}{green!50!black} % Color of a curve in a plot
\colorlet{penColor5}{orange!80!black} % Color of a curve in a plot
\colorlet{fill1}{penColor!20} % Color of fill in a plot
\colorlet{fill2}{penColor2!20} % Color of fill in a plot
\colorlet{fillp}{fill1} % Color of positive area
\colorlet{filln}{penColor2!20} % Color of negative area
\colorlet{fill3}{penColor3!20} % Fill
\colorlet{fill4}{penColor4!20} % Fill
\colorlet{fill5}{penColor5!20} % Fill
\colorlet{gridColor}{gray!50} % Color of grid in a plot

\newcommand{\surfaceColor}{violet}
\newcommand{\surfaceColorTwo}{redyellow}
\newcommand{\sliceColor}{greenyellow}




\pgfmathdeclarefunction{gauss}{2}{% gives gaussian
  \pgfmathparse{1/(#2*sqrt(2*pi))*exp(-((x-#1)^2)/(2*#2^2))}%
}


%%%%%%%%%%%%%
%% Vectors
%%%%%%%%%%%%%

%% Simple horiz vectors
\renewcommand{\vector}[1]{\left\langle #1\right\rangle}


%% %% Complex Horiz Vectors with angle brackets
%% \makeatletter
%% \renewcommand{\vector}[2][ , ]{\left\langle%
%%   \def\nextitem{\def\nextitem{#1}}%
%%   \@for \el:=#2\do{\nextitem\el}\right\rangle%
%% }
%% \makeatother

%% %% Vertical Vectors
%% \def\vector#1{\begin{bmatrix}\vecListA#1,,\end{bmatrix}}
%% \def\vecListA#1,{\if,#1,\else #1\cr \expandafter \vecListA \fi}

%%%%%%%%%%%%%
%% End of vectors
%%%%%%%%%%%%%

%\newcommand{\fullwidth}{}
%\newcommand{\normalwidth}{}



%% makes a snazzy t-chart for evaluating functions
%\newenvironment{tchart}{\rowcolors{2}{}{background!90!textColor}\array}{\endarray}

%%This is to help with formatting on future title pages.
\newenvironment{sectionOutcomes}{}{} 



%% Flowchart stuff
%\tikzstyle{startstop} = [rectangle, rounded corners, minimum width=3cm, minimum height=1cm,text centered, draw=black]
%\tikzstyle{question} = [rectangle, minimum width=3cm, minimum height=1cm, text centered, draw=black]
%\tikzstyle{decision} = [trapezium, trapezium left angle=70, trapezium right angle=110, minimum width=3cm, minimum height=1cm, text centered, draw=black]
%\tikzstyle{question} = [rectangle, rounded corners, minimum width=3cm, minimum height=1cm,text centered, draw=black]
%\tikzstyle{process} = [rectangle, minimum width=3cm, minimum height=1cm, text centered, draw=black]
%\tikzstyle{decision} = [trapezium, trapezium left angle=70, trapezium right angle=110, minimum width=3cm, minimum height=1cm, text centered, draw=black]


\title[Dig-In]{Extreme Values and Critical Points}



\outcome{Define a critical point.}
\outcome{Find critical points.}
\outcome{Define absolute maximum and absolute minimum.}
\outcome{Find the absolute max or min of a continuous function on a closed interval.}
\outcome{Define local maximum and local minimum.}
\outcome{Compare and contrast local and absolute maxima and minima.}
\outcome{Identify situations in which an absolute maximum or minimum is guaranteed.}
\outcome{Classify critical points.}
\outcome{State the First Derivative Test.}
\outcome{Apply the First Derivative Test.}
\outcome{State the Second Derivative Test.}
\outcome{Apply the Second Derivative Test.}
\outcome{Define inflection points.}
\outcome{Find inflection points.}
  


\begin{document}
\begin{abstract}
We use derivatives to help locate extrema.  
\end{abstract}
\maketitle


Whether we are interested in a function as a purely mathematical
object or in connection with some application to the real world, it is
often useful to know what the graph of the function looks like. We can
obtain a good picture of the graph using certain crucial information
provided by derivatives of the function.

\section{Extrema}

Local \textit{extrema} on a function are points on the graph where the
$y$-coordinate is larger (or smaller) than all other $y$-coordinates
on the graph at points ``close to'' $(x,y)$. 

\begin{definition}\hfil\index{maximum/minimum!local}
\begin{enumerate}
\item A function $f$ has a \dfn{local maximum} at $x=a$, if $f(a)\ge
  f(x)$ for every $x$ near $a$.
\item A function $f$ has a \dfn{local minimum} at $x=a$, if $f(a)\le
  f(x)$ for every $x$ near $a$.
\end{enumerate}
A \dfn{local extremum}\index{extremum!local} is either a local
maximum or a local minimum.
\end{definition}

\begin{problem}
  True or false: ``All absolute extrema are also local extrema.''
  \begin{multipleChoice}
    \choice[correct]{true}
    \choice{false}
  \end{multipleChoice}
  \begin{feedback}
    All global extrema are local extrema.
  \end{feedback}
\end{problem}

Points where local maxima and minima occur are quite distinctive on the graph of
a function, and are therefore useful in understanding the shape of the
graph. In many applied problems we want to find the largest or
smallest value that a function achieves (for example, we might want
to find the minimum cost at which some task can be performed) and so
identifying maxima and minima will be useful for applied
problems as well.



\section{Critical points}


If $(x,f(x))$ is a point where $f$ reaches a local maximum or minimum,
and if the derivative of $f$ exists at $x$, then the graph has a
tangent line and the tangent line must be horizontal. This is
important enough to state as a theorem, though we will not prove it.

\begin{theorem}[Fermat's Theorem]\index{Fermat's Theorem}\label{theorem:fermat}
If $f$ has a local extremum at $x=a$ and $f$ is differentiable
at $a$, then $f'(a)=0$.
\end{theorem}

\begin{problem}
  Does Fermat's Theorem say that if $f'(a) = 0$, then $f$ has a local
  extrema at $x=a$?
  \begin{multipleChoice}
    \choice{yes}
    \choice[correct]{no}
  \end{multipleChoice}
  \begin{feedback}
    Consider $f(x) = x^3$, $f'(0) = 0$, but $f$ does not have a local
    maximum or minimum at $x=0$.
  \end{feedback}
\end{problem}


Here is an example of a function and its derivative where Fermat's Theorem applies. 
Consider the plots of $f(x) = x^3-4x^2+3x$ and $f'(x) =
3x^2-8x+3$, and look at the output of the derivative at the same x-values where local extrema occur.
\begin{image}
\begin{tikzpicture}
	\begin{axis}[
            domain=-3:3,
            ymax=3,
            ymin=-3,
            %samples=100,
            axis lines =middle, xlabel=$x$, ylabel=$y$,
            every axis y label/.style={at=(current axis.above origin),anchor=south},
            every axis x label/.style={at=(current axis.right of origin),anchor=west}
          ]
          \addplot [dashed, textColor, smooth] plot coordinates {(.451,0) (.451,.631)}; %% {.451};
          \addplot [dashed, textColor, smooth] plot coordinates {(2.215,-2.113) (2.215,0)}; %% axis{2.215};
          \addplot [very thick, penColor2, smooth] {3*x^2-8*x+3};
          \addplot [very thick, penColor, smooth] {x^3-4*x^2+3*x};
          \node at (axis cs:2.5,-2) [anchor=west] {\color{penColor}$f$};  
          \node at (axis cs:.2,2) [anchor=west] {\color{penColor2}$f'$};
          \addplot[color=penColor2,fill=penColor2,only marks,mark=*] coordinates{(.451,0)};  %% closed hole
          \addplot[color=penColor2,fill=penColor2,only marks,mark=*] coordinates{(2.215,0)};  %% closed hole
          \addplot[color=penColor,fill=penColor,only marks,mark=*] coordinates{(.451,.631)};  %% closed hole
          \addplot[color=penColor,fill=penColor,only marks,mark=*] coordinates{(2.215,-2.113)};  %% closed hole
        \end{axis}
\end{tikzpicture}
%% \caption{A plot of $f(x) = x^3-4x^2+3x$ and $f'(x) = 3x^2-8x+3$.}
%% \label{figure:x^3-4x^2+3x}
\end{image}
However, this is only half of the story! Consider $f(x) = x^{2/3}$ and $f'(x) = \frac{2}{3x^{1/3}}$:
\begin{image}
\begin{tikzpicture}
	\begin{axis}[
            domain=-3:3,
            ymax=2,
            ymin=-2,
            axis lines =middle, xlabel=$x$, ylabel=$y$,
            every axis y label/.style={at=(current axis.above origin),anchor=south},
            every axis x label/.style={at=(current axis.right of origin),anchor=west}
          ]
          \addplot [very thick, penColor2, samples=100, smooth,domain=(-3:-.01)] {-(2/3)*abs(x)^(-1/3)};
          \addplot [very thick, penColor2, samples=100, smooth,domain=(.01:3)] {(2/3)*abs(x)^(-1/3)};
          \addplot [very thick, penColor, smooth,domain=(-3:-.01)] {abs(x)^(2/3)};
          \addplot [very thick, penColor, smooth,domain=(.01:3)] {x^(2/3)};         
          \node at (axis cs:-2,1.7) [anchor=west] {\color{penColor}$f$};  
          \node at (axis cs:2,.7) [anchor=west] {\color{penColor2}$f'$};
        \end{axis}
\end{tikzpicture}
%% \caption{A plot of $f(x) = x^{2/3}$ and $f'(x) = \frac{2}{3x^{1/3}}$.}
%% \label{figure:x^{2/3}}
\end{image}
What do you notice about the value of the derivative at the place where the minimum occurs?

This brings us to our next definition.

\begin{definition}\index{critical point}
  A function has a \dfn{critical point} at an interior point in its domain $x=a$ if 
  \[
  f'(a) = 0\qquad\text{or}\qquad \text{$f'(a)$ does not exist.}
  \]
\end{definition}

\begin{warning} 
When looking for local maximum and minimum points, you are likely to
make two sorts of mistakes: 
\begin{itemize}
\item You may forget that a maximum or minimum can occur where the
  derivative does not exist, so don't forget to check whether the
  derivative exists everywhere. 
\item You might assume that any place that the derivative is zero is a
  local maximum or minimum point, but this is not true, consider the
  plots of $f(x) = x^3$ and $f'(x) = 3x^2$.
\begin{image}
\begin{tikzpicture}
	\begin{axis}[
            domain=-3:3,
            ymax=3,
            ymin=-3,
            axis lines =middle, xlabel=$x$, ylabel=$y$,
            every axis y label/.style={at=(current axis.above origin),anchor=south},
            every axis x label/.style={at=(current axis.right of origin),anchor=west}
          ]
          \addplot [very thick, penColor2, smooth] {3*x^2};
          \addplot [very thick, penColor, smooth] {x^3};         
          \node at (axis cs:1,.9) [anchor=west] {\color{penColor}$f$};  
          \node at (axis cs:-.5,1) [anchor=west] {\color{penColor2}$f'$};
        \end{axis}
\end{tikzpicture}
%% \caption{A plot of $f(x) = x^3$ and $f'(x) = 3x^2$. While $f'(0)=0$,
%%   there is neither a maximum nor minimum at $(0,f(0))$.}
%% \label{figure:x^3}
\end{image}
While $f'(0)=0$, there is neither a maximum nor minimum at $(0,f(0))$.
\end{itemize}
\end{warning}



Since the derivative is zero or undefined at both local maximum and
local minimum points, we need a way to determine which, if either,
actually occurs. The most elementary approach is to test directly
whether the $y$ coordinates near the potential maximum or minimum are
above or below the $y$ coordinate at the point of interest. 

It is not always easy to compute the value of a function at a
particular point. The task is made easier by the availability of
calculators and computers, but they have their own drawbacks: they do
not always allow us to distinguish between values that are very close
together. Nevertheless, because this method is conceptually simple and
sometimes easy to perform, you could consider it.  However, be wary of using this approach as an exam answer- it is difficult to justify, and works only on nice enough functions. It is better if you use some of the methods you will learn over the next few cards.




\begin{example}
Find all local maximum and minimum values for the function 
$f(x)=x^3-x$. 
\begin{explanation} 
Write
\[
\ddx f(x)=\answer[given]{3x^2-1}.
\] 
This is defined everywhere and is zero at $x=\pm \sqrt{3}/3$. Looking
first at $x=\sqrt{3}/3$, we see that 
\[
f(\sqrt{3}/3)=\answer[given]{-2\sqrt{3}/9}.
\] 
Now we test two points on either side of $x=\sqrt{3}/3$, making sure
that neither is farther away than the nearest critical point; since
$\sqrt{3}<3$, $\sqrt{3}/3<1$ and we can use $x=0$ and $x=1$. Since
\[
f(0)=0>-2\sqrt{3}/9\qquad\text{and}\qquad f(1)=0>-2\sqrt{3}/9,
\] 
there must be a local minimum at $x=\answer[given]{\sqrt{3}/3}$.

For $x=-\sqrt{3}/3$, we see that $f(-\sqrt{3}/3)=2\sqrt{3}/9$. This
time we can use $x=0$ and $x=-1$, and we find that $f(-1)=f(0)=0<
2\sqrt{3}/9$, so there must be a local maximum at
$x=\answer[given]{-\sqrt{3}/3}$, see the plot below:
\begin{image}
\begin{tikzpicture}
	\begin{axis}[
            domain=-2:2,
            ymax=2,
            ymin=-2,
            %samples=100,
            axis lines =middle, xlabel=$x$, ylabel=$y$,
            every axis y label/.style={at=(current axis.above origin),anchor=south},
            every axis x label/.style={at=(current axis.right of origin),anchor=west}
          ]
          \addplot [dashed, textColor, smooth] plot coordinates {(-.577,0) (-.577,.385)}; %% {.451};
          \addplot [dashed, textColor, smooth] plot coordinates {(.577,-.385) (.577,0)}; %% axis{2.215};

          \addplot [very thick, penColor2, smooth] {3*x^2-1};
          \addplot [very thick, penColor, smooth] {x^3-x};

          \node at (axis cs:1.2,.3) [anchor=west] {\color{penColor}$f$};  
          \node at (axis cs:-.75,1) [anchor=west] {\color{penColor2}$f'$};

          \addplot[color=penColor2,fill=penColor2,only marks,mark=*] coordinates{(-.577,0)};  %% closed hole
          \addplot[color=penColor2,fill=penColor2,only marks,mark=*] coordinates{(.577,0)};  %% closed hole
          \addplot[color=penColor,fill=penColor,only marks,mark=*] coordinates{(-.577,.385)};  %% closed hole
          \addplot[color=penColor,fill=penColor,only marks,mark=*] coordinates{(.577,-.385)};  %% closed hole
        \end{axis}
\end{tikzpicture}
%%\caption{A plot of $f(x) = x^3-x$ and $f'(x) = 3x^2-1$.}
%%\label{figure:x^3-x}
\end{image}
\end{explanation}
\end{example}


\end{document}
