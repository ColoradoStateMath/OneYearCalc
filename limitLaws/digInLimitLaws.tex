\documentclass{ximera}


\outcome{Calculate limits using the limit laws.}

%\usepackage{todonotes}

\newcommand{\todo}{}

\usepackage{tkz-euclide}
\tikzset{>=stealth} %% cool arrow head
\tikzset{shorten <>/.style={ shorten >=#1, shorten <=#1 } } %% allows shorter vectors

\usetikzlibrary{backgrounds} %% for boxes around graphs
\usetikzlibrary{shapes,positioning}  %% Clouds and stars
\usetikzlibrary{matrix} %% for matrix
\usepgfplotslibrary{polar} %% for polar plots
\usetkzobj{all}
\usepackage[makeroom]{cancel} %% for strike outs
%\usepackage{mathtools} %% for pretty underbrace % Breaks Ximera
\usepackage{multicol}





\usepackage{array}
\setlength{\extrarowheight}{+.1cm}   
\newdimen\digitwidth
\settowidth\digitwidth{9}
\def\divrule#1#2{
\noalign{\moveright#1\digitwidth
\vbox{\hrule width#2\digitwidth}}}





\newcommand{\RR}{\mathbb R}
\newcommand{\R}{\mathbb R}
\newcommand{\N}{\mathbb N}
\newcommand{\Z}{\mathbb Z}

%\renewcommand{\d}{\,d\!}
\renewcommand{\d}{\mathop{}\!d}
\newcommand{\dd}[2][]{\frac{\d #1}{\d #2}}
\newcommand{\pp}[2][]{\frac{\partial #1}{\partial #2}}
\renewcommand{\l}{\ell}
\newcommand{\ddx}{\frac{d}{\d x}}

\newcommand{\zeroOverZero}{\ensuremath{\boldsymbol{\tfrac{0}{0}}}}
\newcommand{\inftyOverInfty}{\ensuremath{\boldsymbol{\tfrac{\infty}{\infty}}}}
\newcommand{\zeroOverInfty}{\ensuremath{\boldsymbol{\tfrac{0}{\infty}}}}
\newcommand{\zeroTimesInfty}{\ensuremath{\small\boldsymbol{0\cdot \infty}}}
\newcommand{\inftyMinusInfty}{\ensuremath{\small\boldsymbol{\infty - \infty}}}
\newcommand{\oneToInfty}{\ensuremath{\boldsymbol{1^\infty}}}
\newcommand{\zeroToZero}{\ensuremath{\boldsymbol{0^0}}}
\newcommand{\inftyToZero}{\ensuremath{\boldsymbol{\infty^0}}}



\newcommand{\numOverZero}{\ensuremath{\boldsymbol{\tfrac{\#}{0}}}}
\newcommand{\dfn}{\textbf}
%\newcommand{\unit}{\,\mathrm}
\newcommand{\unit}{\mathop{}\!\mathrm}
\newcommand{\eval}[1]{\bigg[ #1 \bigg]}
\newcommand{\seq}[1]{\left( #1 \right)}
\renewcommand{\epsilon}{\varepsilon}
\renewcommand{\iff}{\Leftrightarrow}

\DeclareMathOperator{\arccot}{arccot}
\DeclareMathOperator{\arcsec}{arcsec}
\DeclareMathOperator{\arccsc}{arccsc}
\DeclareMathOperator{\si}{Si}
\DeclareMathOperator{\proj}{proj}
\DeclareMathOperator{\scal}{scal}


\newcommand{\tightoverset}[2]{% for arrow vec
  \mathop{#2}\limits^{\vbox to -.5ex{\kern-0.75ex\hbox{$#1$}\vss}}}
\newcommand{\arrowvec}[1]{\tightoverset{\scriptstyle\rightharpoonup}{#1}}
\renewcommand{\vec}{\mathbf}
\newcommand{\veci}{\vec{i}}
\newcommand{\vecj}{\vec{j}}
\newcommand{\veck}{\vec{k}}
\newcommand{\vecl}{\boldsymbol{\l}}

\newcommand{\dotp}{\bullet}
\newcommand{\cross}{\boldsymbol\times}
\newcommand{\grad}{\boldsymbol\nabla}
\newcommand{\divergence}{\grad\dotp}
\newcommand{\curl}{\grad\cross}
%\DeclareMathOperator{\divergence}{divergence}
%\DeclareMathOperator{\curl}[1]{\grad\cross #1}


\colorlet{textColor}{black} 
\colorlet{background}{white}
\colorlet{penColor}{blue!50!black} % Color of a curve in a plot
\colorlet{penColor2}{red!50!black}% Color of a curve in a plot
\colorlet{penColor3}{red!50!blue} % Color of a curve in a plot
\colorlet{penColor4}{green!50!black} % Color of a curve in a plot
\colorlet{penColor5}{orange!80!black} % Color of a curve in a plot
\colorlet{fill1}{penColor!20} % Color of fill in a plot
\colorlet{fill2}{penColor2!20} % Color of fill in a plot
\colorlet{fillp}{fill1} % Color of positive area
\colorlet{filln}{penColor2!20} % Color of negative area
\colorlet{fill3}{penColor3!20} % Fill
\colorlet{fill4}{penColor4!20} % Fill
\colorlet{fill5}{penColor5!20} % Fill
\colorlet{gridColor}{gray!50} % Color of grid in a plot

\newcommand{\surfaceColor}{violet}
\newcommand{\surfaceColorTwo}{redyellow}
\newcommand{\sliceColor}{greenyellow}




\pgfmathdeclarefunction{gauss}{2}{% gives gaussian
  \pgfmathparse{1/(#2*sqrt(2*pi))*exp(-((x-#1)^2)/(2*#2^2))}%
}


%%%%%%%%%%%%%
%% Vectors
%%%%%%%%%%%%%

%% Simple horiz vectors
\renewcommand{\vector}[1]{\left\langle #1\right\rangle}


%% %% Complex Horiz Vectors with angle brackets
%% \makeatletter
%% \renewcommand{\vector}[2][ , ]{\left\langle%
%%   \def\nextitem{\def\nextitem{#1}}%
%%   \@for \el:=#2\do{\nextitem\el}\right\rangle%
%% }
%% \makeatother

%% %% Vertical Vectors
%% \def\vector#1{\begin{bmatrix}\vecListA#1,,\end{bmatrix}}
%% \def\vecListA#1,{\if,#1,\else #1\cr \expandafter \vecListA \fi}

%%%%%%%%%%%%%
%% End of vectors
%%%%%%%%%%%%%

%\newcommand{\fullwidth}{}
%\newcommand{\normalwidth}{}



%% makes a snazzy t-chart for evaluating functions
%\newenvironment{tchart}{\rowcolors{2}{}{background!90!textColor}\array}{\endarray}

%%This is to help with formatting on future title pages.
\newenvironment{sectionOutcomes}{}{} 



%% Flowchart stuff
%\tikzstyle{startstop} = [rectangle, rounded corners, minimum width=3cm, minimum height=1cm,text centered, draw=black]
%\tikzstyle{question} = [rectangle, minimum width=3cm, minimum height=1cm, text centered, draw=black]
%\tikzstyle{decision} = [trapezium, trapezium left angle=70, trapezium right angle=110, minimum width=3cm, minimum height=1cm, text centered, draw=black]
%\tikzstyle{question} = [rectangle, rounded corners, minimum width=3cm, minimum height=1cm,text centered, draw=black]
%\tikzstyle{process} = [rectangle, minimum width=3cm, minimum height=1cm, text centered, draw=black]
%\tikzstyle{decision} = [trapezium, trapezium left angle=70, trapezium right angle=110, minimum width=3cm, minimum height=1cm, text centered, draw=black]

\title[Dig-In:]{The limit laws}
\begin{document}
\begin{abstract}
We give basic laws for working with limits. 
\end{abstract}
\maketitle

In this section, we present a handful of rules called the \textit{Limit Laws}
that allow us to find limits of various combinations of functions.

\begin{theorem}[Limit Laws]\index{limit laws}\label{theorem:limit-laws}
Suppose that $\displaystyle\lim_{x\to a}f(x)=L$, $\displaystyle\lim_{x\to a}g(x)=M$.
\begin{description}
\item[\textbf{Constant Multiple Law}] $\displaystyle\lim_{x\to a} kf(x) = k\displaystyle\lim_{x\to a}f(x)=kL$.\\
\item[Sum/Difference Law] $\displaystyle\lim_{x\to a} (f(x) \pm g(x)) =
  \lim_{x\to a}f(x) \pm \lim_{x\to a}g(x)=L \pm M$.
\item[Product Law]  $\displaystyle\lim_{x\to a} (f(x)g(x)) = \displaystyle\lim_{x\to
  a}f(x)\cdot\lim_{x\to a}g(x)=LM$.
\item[Quotient Law]  $\displaystyle\lim_{x\to a} \frac{f(x)}{g(x)} =
  \frac{\displaystyle\lim_{x\to a}f(x)}{\displaystyle\lim_{x\to a}g(x)}=\frac{L}{M}$, if
  $M\ne0$.
\end{description}
\label{thm:limit laws}
\end{theorem}


\begin{example}
  Compute the following limit using limit laws:
  \[
  \displaystyle\lim_{x\to 1}(5x^2+3x-2)
  \]
\begin{explanation}
  Well, get out your pencil and write with me:
  \[
  \displaystyle\lim_{x\to 1} (5x^2+3x-2) = \lim_{x\to 1} 5x^2 + \lim_{x\to 1} \answer[given]{3x} - \lim_{x\to 1}2
  \]
  by the Sum/Difference Law. So now
  \[
  = 5\displaystyle\lim_{x\to 1} x^2 + 3\lim_{x\to 1} x - \lim_{x\to 1}\answer[given]{2}
  \]
  
  \[
  = 5(1)^2 + 3(1) - 2 =\answer[given]{6}.
  \]
  \begin{prompt}
    We can check our answer by looking at the graph of $y=f(x)$:
    \[
    \graph{5x^2+3x-2}
    \]
  \end{prompt}
\end{explanation}  
\end{example}


The most important thing to learn from this section is whether the
limit laws can be applied for a certain problem, and when we need to
do something more interesting.  We will begin discussing those more
interesting cases in the next section.  For now, we end this section
with a question:

\section{A list of questions}

Let's try this out.

\begin{question}
  Can this limit be directly computed by limit laws?
  \[
  \displaystyle\lim_{x\to 2}\frac{x^2+3x+2}{x+2} 
  \]
  \begin{multipleChoice}
    \choice[correct]{yes}
    \choice{no}
  \end{multipleChoice}
  \begin{question}
    Compute:
    \[
    \displaystyle\lim_{x\to 2}\frac{x^2+3x+2}{x+2}\begin{prompt} =\answer{3}\end{prompt}
    \]
    \begin{feedback}
      Since $f(x)=\frac{x^2+3x+2}{x+2}$ is a rational function, and
      the denominator does not equal $0$, we see that $f(x)$ is
      continuous at $x=2$.  Thus, to find this limit, it suffices to
      plug $2$ into $f(x)$.
    \end{feedback}
  \end{question}
\end{question}


\begin{question}
  Can this limit be directly computed by limit laws?
  \[
  \displaystyle\lim_{x\to 2}\frac{x^2-3x+2}{x-2}
  \]
  \begin{multipleChoice}
    \choice{yes}
    \choice[correct]{no}
  \end{multipleChoice}
  \begin{feedback}
    $f(x) = \frac{x^2-3x+2}{x-2}$ is a rational function, but the
    denominator $x-2$ equals $0$ when $x=2$. None of our current
    theorems address the situation when the denominator of a fraction
    approaches $0$.
  \end{feedback}
\end{question}


\begin{question}
  Can this limit be directly computed by limit laws?
  \[
  \displaystyle\lim_{x\to 0} x\sin(1/x)
  \]
  \begin{multipleChoice}
    \choice{yes}
    \choice[correct]{no}
  \end{multipleChoice}
  \begin{feedback}
    If we are trying to use limit laws to compute this limit, we would
    first have to use the Product Law to say that
    \[
    \displaystyle\lim_{x\to 0}x\sin(1/x)= \lim_{x\to 0} x \cdot \lim_{x\to 0} \sin(1/x).
    \]
    We are only allowed to use this law if both limits exist, so we
    must check this first.  We know from continuity that
    \[
    \displaystyle\lim_{x\to  0}x=0.
    \]
    However, we also know that $\sin(1/x)$ oscillates ``wildly'' as
    $x$ approaches $0$, and so the limit
    \[
    \lim_{x\to 0} \sin(1/x)
    \]does not exist.  Therefore, we cannot use the
    Product Law.
  \end{feedback}
\end{question}


\begin{question}
  Can this limit be directly computed by limit laws?
  \[
  \displaystyle\lim_{x\to 0} \cot(x^3)
  \]
  \begin{multipleChoice}
    \choice{yes}
    \choice[correct]{no}
  \end{multipleChoice}
  \begin{feedback}
    Notice that
    \[
    \cot(x^3) = \frac{\cos(x^3)}{\sin(x^3)}.
    \]
    If we are trying to use limit laws to compute this limit, we would
    like to use the Quotient Law to say that
    \[
    \displaystyle\lim_{x\to 0} \frac{\cos(x^3)}{\sin(x^3)} = \displaystyle\frac{\lim_{x\to 0}
      \cos(x^3)}{\lim_{x\to 0} \sin(x^3)}.
    \]
    We are only allowed to use this law if both limits exist and the
    denominator is not $0$. We suspect that the limit on on the
    denominator might equal $0$, so we check this limit.
    \begin{align*}
      \displaystyle\lim_{x\to 0} \sin(x^3) &= \displaystyle\sin(\lim_{x\to 0}x^3)\\
      &=\sin(0) \\
      &=0.
  \end{align*}
  This means that the denominator is zero and hence we cannot use the
  Quotient Law.
  \end{feedback}
\end{question}


\begin{question}
  Can this limit be directly computed by limit laws?
  \[
  \displaystyle\lim_{x\to 1}\sec^2(\sqrt{x}-1)
  \]
  \begin{multipleChoice}
    \choice[correct]{yes}
    \choice{no}
  \end{multipleChoice}
  \begin{question}
    Compute:
    \[
    \displaystyle\lim_{x\to 1}\sec^2(\sqrt{x}-1)\begin{prompt} =\answer{1}\end{prompt}
    \]
    \begin{feedback}
      Notice that
      \[
      \displaystyle\lim_{x\to 1} \sec^2(\sqrt{x}-1) = \lim_{x\to 1} \frac{1}{\cos^2(\sqrt{x}-1)}.
      \]
      If we are trying to use Limit Laws to compute this limit, we
      would now have to use the Quotient Law to say that
      \[
      \displaystyle\lim_{x\to 1} \frac{1}{\cos^2(\sqrt{x}-1)} = \frac{ \lim_{x\to 1}1}{
        \lim_{x\to 1}\cos^2(\sqrt{x}-1)}.
      \]
      We are only allowed to use this law if both limits exist and the
      denominator is not $0$.  Let's check the denominator and numerator
      separately. First we'll compute the limit of the denominator:
      \begin{align*}
        \displaystyle\lim_{x\to 1}\cos^2(\sqrt{x}-1) &= \cos^2(\lim_{x\to 1}(\sqrt{x}-1))\\
        &=\cos^2(\lim_{x\to 1}(\sqrt{x})-\lim_{x\to 1}(1))\\
        &=\cos^2(1-1)\\
        &= \cos^2(0)\\
        &=1
      \end{align*}
      Therefore, the limit in the denominator exists and does not
      equal $0$. We can use the Quotient Law, so we will compute the limit of the numerator:
      \[
      \lim_{x\to 1}1=1
      \]
      Hence
      \[
      \frac{ \lim_{x\to 1}1}{ \lim_{x\to 1}\cos^2(\sqrt{x}-1)} =
      \frac{1}{1}=1
      \]
    \end{feedback}
  \end{question}
\end{question}


\begin{question}
  Can this limit be directly computed by limit laws?
  \[
  \displaystyle\lim_{x\to 4}{\left(\frac{2x}{x-4}-\frac{8}{x-4}\right)}
  \]
  \begin{multipleChoice}
    \choice{yes}
    \choice[correct]{no}
  \end{multipleChoice}
  \begin{feedback}
    If we are trying to use limit laws to compute this limit, we would have to use the Product Law to say that
    \[
    \displaystyle\lim_{x\to 4}\left(\frac{2x}{x-4} - \frac{8}{x-4}\right)= \displaystyle\lim_{x\to 4}\frac{2x}{x-4} - \lim_{x\to 4}\frac{8}{x-4}.
    \]
    We are only allowed to use this law if both limits exist.  Let's
    check each limit separately.
    \begin{align*}
      \displaystyle\lim_{x\to 4}\frac{2x}{x-4}=\frac{\lim_{x\to 4}2x}{\lim_{x\to 4}(x-4)}
    \end{align*}

and

    \begin{align*}
      \displaystyle\lim_{x\to 4}\frac{8}{x-4}=\displaystyle\frac{\lim_{x\to 4}8}{\lim_{x\to 4}(x-4)}
    \end{align*}
    
   We are only allowed to use this law if both limits exist. The limits in the numerators definitely
   exist. However, the denominator is equal to $0$ for both limits. Therefore, we cannot use the limit laws. 
   
   We can use algebra and simplify.
   
   \begin{align*}
   \displaystyle\lim_{x\to 4}\frac{2x-8}{x-4}&=\lim_{x\to 4}\frac{2(x-4)}{x-4}\\
   &=\lim_{x\to 4}2\\
   &=2
  \end{align*}
  Since the denominator is $0$, we cannot apply the Quotient Law.
  \end{feedback}
\end{question}

\begin{question}
  Can this limit be directly computed by limit laws?
  \[
  \displaystyle\lim_{x\to 0} \cot(x)\sin(x)
  \]
  \begin{multipleChoice}
    \choice{yes}
    \choice[correct]{no}
  \end{multipleChoice}
  \begin{feedback}
  If we are trying to use limit laws to compute this limit, we would
  have to use the Product Law to say that
  \[
  \displaystyle\lim_{x\to 0} \cot(x)\sin(x) =\lim_{x\to 0} \cot(x) \cdot \lim_{x\to 0}\sin(x).
  \]
  We are only allowed to use this law if both limits exist.  We know
  $\displaystyle\lim_{x\to 0} \sin(x) = 0$, but what about $\displaystyle\lim_{x\to 0}\cot(x)$?  We do
  not know how to find $\lim_{x\to 0}\cot(x)$ using limit laws because $0$
  is not in the domain of $\cot(x)$.
  \end{feedback}
  
  However, see what you can do with trigonometric identities. Try rewriting $\cot(x)$ as $\frac{\cos(x)}{\sin(x)}$
\end{question}


\begin{question}
  Can this limit be directly computed by limit laws?
  \[
  \displaystyle\lim_{x\to 0} \frac{\sin(x)}{1+x}
  \]
  \begin{multipleChoice}
    \choice[correct]{yes}
    \choice{no}
  \end{multipleChoice}
  \begin{question}
    Compute:
    \[
    \displaystyle\lim_{x\to 0} \frac{\sin(x)}{1+x}\begin{prompt} =\answer{0}\end{prompt}
    \]
    \begin{feedback}
      If we are trying to use limit laws to compute this limit, we
      would have to use the Quotient Law to say that
      \[
      \displaystyle\lim_{x\to 0} \frac{\sin(x)}{1+x} = \displaystyle\frac{\lim_{x\to
          0}\sin(x)}{\lim_{x\to 0}(1+x)}.
      \]
      We are only allowed to use this law if both limits exist and the
      denominator does not equal $0$.  Let's check each limit
      separately, starting with the denominator
      
      The limits in both the numerator and denominator exist and the
      limit in the denominator does not equal $0$, so we can use the
      Quotient Law.  We find:
      \[
        \displaystyle \frac{\lim_{x\to 0}\sin(x)}{\lim_{x\to 0}(1+x)}=\frac{0}{1}=0.
        \]
    \end{feedback}
  \end{question}
\end{question}


\begin{question}
  Can this limit be directly computed by limit laws?
  \[
  \displaystyle\lim_{x\to 0}(1+x)^{1/x}
  \]
  \begin{multipleChoice}
    \choice{yes}
    \choice[correct]{no}
  \end{multipleChoice}
  \begin{feedback}
  We do not have any limit laws for functions of the form $f(x)^{g(x)}$, so we cannot compute this limit.
  \end{feedback}
\end{question}

\end{document}
