\documentclass[handout]{ximera}
%\documentclass[10pt,handout,twocolumn,twoside,wordchoicegiven]{xercises}
%\documentclass[10pt,handout,twocolumn,twoside,wordchoicegiven]{xourse}

%\author{Brady Tyburski}
%\license{Creative Commons 3.0 By-NC}
\input{../preamble.tex}

\outcome{Practice on Areas and Riemann Sums.}
   

\title[Exercises:]{Areas and Riemann Sums Exercises}

\begin{document}
\begin{abstract}
  Here we'll practice Riemann Sums. If the answer is false, be sure to have a counter example prepared!
\end{abstract}
\maketitle

%%AreaTF1
\begin{exercise}
 The midpoint method for approximating area under a curve will always give a more accurate result than the right or left endpoint methods.  
 	\begin{multipleChoice}	
		\choice{True}
		\choice[correct]{False}
	\end{multipleChoice}
\end{exercise}

%%AreaTF1
\begin{exercise}
$S_{10}$ will \textit{always} be a better estimate than $S_{5}$.  
	\begin{multipleChoice}	
		\choice{True}
		\choice[correct]{False}
	\end{multipleChoice}
\end{exercise}

%%AreaTF1
\begin{exercise}
A Riemann sum must always be positive.  
	\begin{multipleChoice}	
		\choice{True}
		\choice[correct]{False}
	\end{multipleChoice}
\end{exercise}

%%AreaTF1
\begin{exercise}
A left Riemann sum is never equal to a right Riemann sum. 
	\begin{multipleChoice}	
		\choice{True}
		\choice[correct]{False}
	\end{multipleChoice}
\end{exercise}

%%AreaTF1
\begin{exercise}
It is impossible for a lower sum to be equal to an upper sum.  
	\begin{multipleChoice}	
		\choice{True}
		\choice[correct]{False}
	\end{multipleChoice}
\end{exercise}

%%AreaTF1
\begin{exercise}
$\sum\limits_{k=0}^3 3k = \sum\limits_{n=0}^{3} 3n$
	\begin{multipleChoice}	
		\choice[correct]{True}
		\choice{False}
	\end{multipleChoice}
\end{exercise}

%%AreaTF1
\begin{exercise}
It is possible to express the sum $2+4+6+8+10$ more than one way with sigma notation.  
	\begin{multipleChoice}	
		\choice[correct]{True}
		\choice{False}
	\end{multipleChoice}
\end{exercise}

%%AreaTF1
\begin{exercise}
$\sum\limits_{k=0}^2 k^2 = \sum\limits_{k=-1}^{1} 2k^2$
	\begin{multipleChoice}	
		\choice{True}
		\choice[correct]{False}
	\end{multipleChoice}
\end{exercise}







\end{document}
