\documentclass[handout]{ximera}
%\documentclass[10pt,handout,twocolumn,twoside,wordchoicegiven]{xercises}
%\documentclass[10pt,handout,twocolumn,twoside,wordchoicegiven]{xourse}

%\author{Steven Gubkin}
%\license{Creative Commons 3.0 By-NC}
%\usepackage{todonotes}

\newcommand{\todo}{}

\usepackage{tkz-euclide}
\tikzset{>=stealth} %% cool arrow head
\tikzset{shorten <>/.style={ shorten >=#1, shorten <=#1 } } %% allows shorter vectors

\usetikzlibrary{backgrounds} %% for boxes around graphs
\usetikzlibrary{shapes,positioning}  %% Clouds and stars
\usetikzlibrary{matrix} %% for matrix
\usepgfplotslibrary{polar} %% for polar plots
\usetkzobj{all}
\usepackage[makeroom]{cancel} %% for strike outs
%\usepackage{mathtools} %% for pretty underbrace % Breaks Ximera
\usepackage{multicol}





\usepackage{array}
\setlength{\extrarowheight}{+.1cm}   
\newdimen\digitwidth
\settowidth\digitwidth{9}
\def\divrule#1#2{
\noalign{\moveright#1\digitwidth
\vbox{\hrule width#2\digitwidth}}}





\newcommand{\RR}{\mathbb R}
\newcommand{\R}{\mathbb R}
\newcommand{\N}{\mathbb N}
\newcommand{\Z}{\mathbb Z}

%\renewcommand{\d}{\,d\!}
\renewcommand{\d}{\mathop{}\!d}
\newcommand{\dd}[2][]{\frac{\d #1}{\d #2}}
\newcommand{\pp}[2][]{\frac{\partial #1}{\partial #2}}
\renewcommand{\l}{\ell}
\newcommand{\ddx}{\frac{d}{\d x}}

\newcommand{\zeroOverZero}{\ensuremath{\boldsymbol{\tfrac{0}{0}}}}
\newcommand{\inftyOverInfty}{\ensuremath{\boldsymbol{\tfrac{\infty}{\infty}}}}
\newcommand{\zeroOverInfty}{\ensuremath{\boldsymbol{\tfrac{0}{\infty}}}}
\newcommand{\zeroTimesInfty}{\ensuremath{\small\boldsymbol{0\cdot \infty}}}
\newcommand{\inftyMinusInfty}{\ensuremath{\small\boldsymbol{\infty - \infty}}}
\newcommand{\oneToInfty}{\ensuremath{\boldsymbol{1^\infty}}}
\newcommand{\zeroToZero}{\ensuremath{\boldsymbol{0^0}}}
\newcommand{\inftyToZero}{\ensuremath{\boldsymbol{\infty^0}}}



\newcommand{\numOverZero}{\ensuremath{\boldsymbol{\tfrac{\#}{0}}}}
\newcommand{\dfn}{\textbf}
%\newcommand{\unit}{\,\mathrm}
\newcommand{\unit}{\mathop{}\!\mathrm}
\newcommand{\eval}[1]{\bigg[ #1 \bigg]}
\newcommand{\seq}[1]{\left( #1 \right)}
\renewcommand{\epsilon}{\varepsilon}
\renewcommand{\iff}{\Leftrightarrow}

\DeclareMathOperator{\arccot}{arccot}
\DeclareMathOperator{\arcsec}{arcsec}
\DeclareMathOperator{\arccsc}{arccsc}
\DeclareMathOperator{\si}{Si}
\DeclareMathOperator{\proj}{proj}
\DeclareMathOperator{\scal}{scal}


\newcommand{\tightoverset}[2]{% for arrow vec
  \mathop{#2}\limits^{\vbox to -.5ex{\kern-0.75ex\hbox{$#1$}\vss}}}
\newcommand{\arrowvec}[1]{\tightoverset{\scriptstyle\rightharpoonup}{#1}}
\renewcommand{\vec}{\mathbf}
\newcommand{\veci}{\vec{i}}
\newcommand{\vecj}{\vec{j}}
\newcommand{\veck}{\vec{k}}
\newcommand{\vecl}{\boldsymbol{\l}}

\newcommand{\dotp}{\bullet}
\newcommand{\cross}{\boldsymbol\times}
\newcommand{\grad}{\boldsymbol\nabla}
\newcommand{\divergence}{\grad\dotp}
\newcommand{\curl}{\grad\cross}
%\DeclareMathOperator{\divergence}{divergence}
%\DeclareMathOperator{\curl}[1]{\grad\cross #1}


\colorlet{textColor}{black} 
\colorlet{background}{white}
\colorlet{penColor}{blue!50!black} % Color of a curve in a plot
\colorlet{penColor2}{red!50!black}% Color of a curve in a plot
\colorlet{penColor3}{red!50!blue} % Color of a curve in a plot
\colorlet{penColor4}{green!50!black} % Color of a curve in a plot
\colorlet{penColor5}{orange!80!black} % Color of a curve in a plot
\colorlet{fill1}{penColor!20} % Color of fill in a plot
\colorlet{fill2}{penColor2!20} % Color of fill in a plot
\colorlet{fillp}{fill1} % Color of positive area
\colorlet{filln}{penColor2!20} % Color of negative area
\colorlet{fill3}{penColor3!20} % Fill
\colorlet{fill4}{penColor4!20} % Fill
\colorlet{fill5}{penColor5!20} % Fill
\colorlet{gridColor}{gray!50} % Color of grid in a plot

\newcommand{\surfaceColor}{violet}
\newcommand{\surfaceColorTwo}{redyellow}
\newcommand{\sliceColor}{greenyellow}




\pgfmathdeclarefunction{gauss}{2}{% gives gaussian
  \pgfmathparse{1/(#2*sqrt(2*pi))*exp(-((x-#1)^2)/(2*#2^2))}%
}


%%%%%%%%%%%%%
%% Vectors
%%%%%%%%%%%%%

%% Simple horiz vectors
\renewcommand{\vector}[1]{\left\langle #1\right\rangle}


%% %% Complex Horiz Vectors with angle brackets
%% \makeatletter
%% \renewcommand{\vector}[2][ , ]{\left\langle%
%%   \def\nextitem{\def\nextitem{#1}}%
%%   \@for \el:=#2\do{\nextitem\el}\right\rangle%
%% }
%% \makeatother

%% %% Vertical Vectors
%% \def\vector#1{\begin{bmatrix}\vecListA#1,,\end{bmatrix}}
%% \def\vecListA#1,{\if,#1,\else #1\cr \expandafter \vecListA \fi}

%%%%%%%%%%%%%
%% End of vectors
%%%%%%%%%%%%%

%\newcommand{\fullwidth}{}
%\newcommand{\normalwidth}{}



%% makes a snazzy t-chart for evaluating functions
%\newenvironment{tchart}{\rowcolors{2}{}{background!90!textColor}\array}{\endarray}

%%This is to help with formatting on future title pages.
\newenvironment{sectionOutcomes}{}{} 



%% Flowchart stuff
%\tikzstyle{startstop} = [rectangle, rounded corners, minimum width=3cm, minimum height=1cm,text centered, draw=black]
%\tikzstyle{question} = [rectangle, minimum width=3cm, minimum height=1cm, text centered, draw=black]
%\tikzstyle{decision} = [trapezium, trapezium left angle=70, trapezium right angle=110, minimum width=3cm, minimum height=1cm, text centered, draw=black]
%\tikzstyle{question} = [rectangle, rounded corners, minimum width=3cm, minimum height=1cm,text centered, draw=black]
%\tikzstyle{process} = [rectangle, minimum width=3cm, minimum height=1cm, text centered, draw=black]
%\tikzstyle{decision} = [trapezium, trapezium left angle=70, trapezium right angle=110, minimum width=3cm, minimum height=1cm, text centered, draw=black]


\outcome{Exercises: Second Derivative Test}
   

\title{Second Derivative Test Exercises}

\begin{document}
\begin{abstract}
  Here we'll practice using the second derivative test.
\end{abstract}
\maketitle


%%PROBLEM 6
\begin{exercise}

If $f''(x)<0$ on $(2,5)$, and $f'(3)=0$, then $f(3)$ is a local max
	\begin{multipleChoice}	
		\choice[correct]{True}
		\choice{False}
	\end{multipleChoice}
\end{exercise}

%%PROBLEM 1
\begin{exercise}
The function $f(x) = x^3-27x+6$ has two critical points.  If we call
these critical points $a$ and $b$, and order them such that $a < b$,
then

$$
a = \answer{-3}
$$

$$
b=\answer{3}
$$

$f''(a)$ is \wordChoice{\choice{positive} \choice[correct]{negative}}, so $f(a)$ is a local \wordChoice{\choice[correct]{max} \choice{min}}.

$f''(b)$ is \wordChoice{\choice[correct]{positive} \choice{negative}}, so $f(b)$ is a local \wordChoice{\choice{max} \choice[correct]{min}}.
\end{exercise}

%%PROBLEM 2
\begin{exercise}
The function $f(x) = \displaystyle\frac{x}{1+x^2}$ has two critical points.  If we call
these critical points $a$ and $b$, and order them such that $a < b$,
then

$$
a = \answer{-1}
$$

$$
b=\answer{1}
$$

$f''(a)$ is \wordChoice{\choice[correct]{positive} \choice{negative}}, so $f(a)$ is a local \wordChoice{\choice{max} \choice[correct]{min}}.

$f''(b)$ is \wordChoice{\choice{positive} \choice[correct]{negative}}, so $f(b)$ is a local \wordChoice{\choice[correct]{max} \choice{min}}.
\end{exercise}

%%PROBLEM 3
\begin{exercise}
Consider the function $f(x) = x\sqrt{4-x}$.

What is the domain of $f(x)$?  Domain: $(-\infty,  \answer{4}]$

$f(x)$ has only one critical point, $x=a$.

$$
a = \answer{8/3}
$$


$f''(a)$ is \wordChoice{\choice{positive} \choice[correct]{negative}}, so $f(a)$ is a local \wordChoice{\choice[correct]{max} \choice{min}}.
\end{exercise}

%%PROBLEM 4
\begin{exercise}
The function $f(x) = x^4-4x^3+16x-3$ has two critical points. (You may wish to use a calculator to find these critical points.) If we
call these critical points $a$ and $b$, and order them such that $a <
b$, then

$$
a = \answer{-1}
$$

$$
b=\answer{2}
$$

At $x=a$, the second derivative test
\begin{multipleChoice}
\choice{Indicates a local maxima}
\choice[correct]{Indicates a local minima}
\choice{Fails, but the First derivative test indicates a local max}
\choice{Fails, but the First derivative test indicates a local min}
\choice{Fails, and the First derivative test indicates that it is not a local extrema}
\end{multipleChoice}

At $x=b$, the second derivative test
\begin{multipleChoice}
\choice{Indicates a local maxima}
\choice{Indicates a local minima}
\choice{Fails, but the First derivative test indicates a local max}
\choice{Fails, but the First derivative test indicates a local min}
\choice[correct]{Fails, and the First derivative test indicates that it is not a local extrema}
\end{multipleChoice}

\end{exercise}

%%PROBLEM 5
\begin{exercise}
The function $f(x) = x^4-4 x^3+6 x^2-4 x+3$ has only one critical point, $x=a$.  You may wish to use technology to find this critical point.

$$
a = \answer{1}
$$

At $x=a$, the second derivative test
\begin{multipleChoice}
\choice{Indicates a local maxima}
\choice{Indicates a local minima}
\choice{Fails, but the First derivative test indicates a local max}
\choice[correct]{Fails, but the First derivative test indicates a local min}
\choice{Fails, and the First derivative test indicates that it is not a local extrema}
\end{multipleChoice}
\end{exercise}




\end{document}
