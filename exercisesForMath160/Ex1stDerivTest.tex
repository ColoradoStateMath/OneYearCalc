\documentclass[handout]{ximera}
%\documentclass[10pt,handout,twocolumn,twoside,wordchoicegiven]{xercises}
%\documentclass[10pt,handout,twocolumn,twoside,wordchoicegiven]{xourse}

%\author{Steven Gubkin}
%\license{Creative Commons 3.0 By-NC}
\input{../preamble.tex}

\outcome{Exercises: First Derivative Test}
   

\title{First Derivative Test Exercises}

\begin{document}
\begin{abstract}
  Here we'll practice using the first derivative test.
\end{abstract}
\maketitle

%%PROBLEM 1
\begin{exercise}
The function $f(x) = x^3-6x+1$ has two critical points.  If we call these critical point $a$ and $b$, and order them such that $a < b$, then

$$
a = \answer{-\sqrt{2}}
$$

$$
b=\answer{\sqrt{2}}
$$

On $(\infty,a)$, $f$ is \wordChoice{\choice[correct]{increasing} \choice{decreasing}}

On $(a,b)$, $f$ is \wordChoice{\choice{increasing} \choice[correct]{decreasing}}

On $(b,\infty)$, $f$ is \wordChoice{\choice[correct]{increasing} \choice{decreasing}}


Consider $f$ restricted to the domain $[-3,2.5]$.

On this interval, the absolute maximum value of $f$ is $\answer{4\sqrt{2}+1}$.

On this interval, the absolute minimum value of $f$ is $\answer{-8}$.

\end{exercise}


%%PROBLEM 2
\begin{exercise}
The function $f(x) = 3x^4-8x^3+6x^2+7$ has two critical points.  If we
call these critical points $a$ and $b$, and order them such that $a <
b $, then

$$
a = \answer{0}
$$

$$
b=\answer{1}
$$



On $(-\infty,a)$, $f$ is \wordChoice{\choice{increasing} \choice[correct]{decreasing}}

On $(a,b)$, $f$ is \wordChoice{\choice[correct]{increasing} \choice{decreasing}}

On $(b,\infty)$, $f$ is \wordChoice{\choice[correct]{increasing} \choice{decreasing}}


$f(a)$ is a \wordChoice{\choice{Local max} \choice[correct]{Local Min} \choice{Neither}}

$f(b)$ is a \wordChoice{\choice{Local max} \choice{Local Min} \choice[correct]{Neither}}


\end{exercise}


%%PROBLEM 3
\begin{exercise}
Consider the function $f(x) =\displaystyle\frac{1}{4\sqrt{x}}+x$.

What is the domain of this function? 

Domain: $ (\answer{0}, \infty)$

$f(x)$ has only one critical point.  Call this critical point $a$ .

$$
a = \answer{0.25}
$$

On $(-\infty,0)$, $f$ is \wordChoice{\choice{increasing} \choice{decreasing} \choice[correct]{not defined}}

On $(0,a)$, $f$ is \wordChoice{\choice{increasing} \choice[correct]{decreasing}}

On $(a,\infty)$, $f$ is \wordChoice{\choice[correct]{increasing} \choice{decreasing}}

$f(a)$ is an absolute \wordChoice{\choice{maximum} \choice[correct]{minimum}}

\end{exercise}

%%PROBLEM 4
\begin{exercise}
The function $f(x) =x+\cos(x)$ has one critical point on the interval $[0,\pi/2]$. Call this critical point $a$.

$$
a = \answer{\pi/6}
$$

On $[0,a)$, $f$ is \wordChoice{\choice[correct]{increasing} \choice{decreasing} }

On $(a,\pi/2]$, $f$ is \wordChoice{\choice[correct]{increasing} \choice{decreasing}}

$f(a)$ is an absolute \wordChoice{\choice{maximum} \choice{minimum} \choice[correct]{neither}}

\end{exercise}

%%Problem 5

\begin{exercise}
The function $f(x)=\displaystyle{\frac{2x-3}{x^2-9}}$ has two critical points.  If we
call these critical points $a$ and $b$, and order them such that $a <
b $, then

$$
a = \answer{0}
$$

$$
b=\answer{1}
$$

On $(-\infty,a)$, $f$ is \wordChoice{\choice{increasing} \choice[correct]{decreasing} }

On $(a,b)$, $f$ is \wordChoice{\choice{increasing} \choice[correct]{decreasing} }

On $(b,\infty)$, $f$ is \wordChoice{\choice{increasing} \choice[correct]{decreasing} }

If the interval is $[-2,2]$, the absolute maximum value is $\answer{1.4}$ and occurs when $x = \answer{-2}$.

\vspace{.25in}

If the interval is $[0,5]$, the absolute maximum value is \wordChoice{\choice{f(0)} \choice{f(b)}  \choice{f(5)} \choice[correct]{f(x) has no absolute maximum.}}


\end{exercise}



















\end{document}